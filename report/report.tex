%----------------------------------------------------------------------------------------
%	PACKAGES AND DOCUMENT CONFIGURATIONS
%----------------------------------------------------------------------------------------

\documentclass[
	letterpaper, % Paper size, specify a4paper (A4) or letterpaper (US letter)
	10pt, % Default font size, specify 10pt, 11pt or 12pt
]{class}

\usepackage{caption}

\addbibresource{bibliography.bib} % Bibliography file (located in the same folder as the template)

%----------------------------------------------------------------------------------------
%	REPORT INFORMATION
%----------------------------------------------------------------------------------------

\title{Airline Departure\\Data Analysis and Regression} % Report title

\author{Lucchi Manuele \& Tricella Davide} % Author name(s), add additional authors like: '\& James \textsc{Smith}'

\date{\today} % Date of the report

%----------------------------------------------------------------------------------------

\begin{document}

\maketitle % Insert the title, author and date using the information specified above

\begin{center}
	\begin{tabular}{l r}
		Instructors: Professor \textsc{Cesa-Bianchi} \& Professor \textsc{Malchiodi}
	\end{tabular}
\end{center}

%----------------------------------------------------------------------------------------
%	DECLARATION
%----------------------------------------------------------------------------------------

\textit{We declare that this material,
	which we now submit for assessment, is entirely our own work and has not been
	taken from the work of others, save and to the extent that such work has been cited and
	acknowledged within the text of our work. We understand that plagiarism, collusion,
	and copying are grave and serious offences in the university and accept the penalties that
	would be imposed should I engage in plagiarism, collusion or copying. This assignment,
	or any part of it, has not been previously submitted by us or any other person for
	assessment on this or any other course of study.}

\tableofcontents

%----------------------------------------------------------------------------------------
%	ABSTRACT
%----------------------------------------------------------------------------------------

\begin{abstract}
	The purpose of this paper is to evaluate the usage of a Logistic Regression model on a airlines dataset to predict flight cancellation or diversion, in a scalable and time/space efficient implementation.
\end{abstract}

\section{Definitions}\label{definitions} % Labels provide a point for referencing, in this case with \ref{definitions} to refer to this subsection number

\begin{description}
	\item[Label]
	\item[Model]
\end{description}

%----------------------------------------------------------------------------------------
%	DATASET
%----------------------------------------------------------------------------------------

\section{Dataset}

The initial dataset, [Airline Delay and Cancellation Data] CITAZIONE is made of 9 years of airlines flights data, composed by 10 files (one for each year from 2009 to 2018) of around 6 milions records each.
The files presents 28 columns, of which we only took the 9 more relevant\\

\begin{description}
	\item[FL\_DATE] The flight date.
	\item[OP\_CARRIER] The carrier code.
	\item[ORIGIN] The departure airport.
	\item[DEST] The destination airport.
	\item[CRS\_DEP\_TIME] The planned departure time.
	\item[CRS\_ARR\_TIME] The planned arrival time.
	\item[CANCELLED] If the flight has been canceled.
	\item[DIVERTED] If the flight has been diverted.
	\item[CRS\_ELAPSED\_TIME] The planned total time of the flight, taxi time included.
	\item[DISTANCE] The distance the flight has to cover.\\
\end{description}

The majority of columns have been excluded because contained information not available at departure time, like the ones regarding actual departure, flight and arrival time, which are at disposal only after the aircraft landed.
Other columns also contained informations which do not have any correlation with the results of the experiments, like the flight number assigned by the carrier.\\

In the case the prediction is about the cancellation, the DIVERTED column will be ignored, while if the prediction is on if the flight would be diverted or not, the CANCELLED column will be ignored.\\
The carrier code is a two characters alphanumeric code, the origin and destination places are a three characters alphanumeric code.\\
Flight date, departure time and arrival time are dates, while the elapsed time and the distance are real numbers.\\
Cancelled and diverted are either 0 or 1.\\

One milion of records equally distributed between the files were taken to perform the training.

%----------------------------------------------------------------------------------------
%	PREPROCESSING
%----------------------------------------------------------------------------------------

\section{Preprocessing Techniques}

\subsection{Algorithms and Techniques}

Multiple preprocessing techniques were used.\\

Initially the rows without valid key values are removed, while the ones with invalid data the columns for which is possible to assign a default value are eventually corrected.\\

Then the data not already represented as real numbers has been converted; airports and carriers identifiers, that were alphanumeric codes, had a number assigned based on the code. Dates were splitted between the year and the rest, the former has been discarded, while the latter has been hashed.\\

Particularly, to convert the identifiers, the checksum function crc32 has been used, then the result has been put in END with the bytes representing the number -1, to ensure to get an unsigned integer.Finally, hte value is normnalized dividing it for the max integer value.
This function has been choosen because is one of the fastest way to hash short strings, which is what is needed here, compared to other alghoritims like the SHA or MD5.\\

To normalize dates, the day of the year has been extracted from every date, then divided by 365. A similar strategy has been used for the departure and arrival times, exctracting the minutes of the day and then dividing by 1140.\\

The distance has been normalized by dividing it for the maximum value found in the dataset, rounded by excess to 4970.\\

The data (now completely composed of real numbers) was then normalized between 0 and 1, to avoid exploding values.\\

At this point, the dataset has been balanced in regard of the evaluated property, be it being canceled or diverted, so that there are an equal number of uniformly drawn positives and negatives.\\

This was necessary since the diverted or cancelled flights are a really small percentage of the overall flights,
this in the first tests has been proved to be a problem, because the trained model always responded that no flight would have been cancelled or diverted, since it hass been trained on a dataset with basically zero problematic flights.
To solve the problem an Oversampling has been applied, matching the number of normal flights and problematic flights.\\

Lastly, the data was splitted between the training set (75\%) and the test set (25\%).\\

DIVISO PER MEDIA E VARIANZA??? L2 REGULARIZATION

\subsection{Parallelization}

Keeping in mind that the implementation has to be space and time efficient and scale up to large datasets, the preprocessing part has been carried out using the library PySpark.
PySpark is a wrapper for Python of the library Apache Spark, originally written in Java.\\

The purpose of this library is the handling of parallelized data processing, particularly regarding the Distributed File System Hadoop, also created by the Apache Foundation.
The library handles automatically the work distribution on the available nodes that the system provide, it can be composed of a single machine with multiple cores or a cluster of machines, this improves significantly the scalability of the solution, which can be run on competely different system without code modification.\\

The usage of this library created some compatibility issues, because the data structures used by PySpark were not compatible with various parts of the preprocessing section, which had been written initially suing the data science library Pandas.
To solve these problems, it wasn't possible to simply use a conversion and leave the parts written in Pandas as they were, because the computation would have run on a single machine, without parallelization, making the use of PySpark completely pointless.
The issue has been addressed using the PySpark.SQL functionality, which allow to execute queries on a distributed dataframe. For our purposes various UDF(User Defined Functions), have been created, which then have been applied to every column containing certain types of data.\\

The library PySpark is also able to handle the csv file reading and writing, so it has been used to save the preprocessed data to speed up multiple runs on the dataset. To carry out the writing of the various distributed dataframes, various files are created, then at the time of reading, the data is distributed to the various nodes.\\

The preprocessing part of the solution is therefore computed in a distributed manner, using the PySpark dataframe as main data structure to perform the various calculations on the columns of the dataset.
At the end of this section of the solution, the distributed dataframes are merged into one using the collect method. This operation can create memory problems if the dataset is extremely large, but it is not possibile to distribute efficiently the model training
and computing as easily as the preprocessing, so the main part of the solution will be computed using standard Python data structures.

%----------------------------------------------------------------------------------------
%	MODEL
%----------------------------------------------------------------------------------------

\section{Model}

\subsection{Parameters initialization}
Parameters such as Weights and Bias are initialized using a \textbf{uniform distribution} between 0 and 1, with the first one having the same length as the number of columns and the second being a scalar value.

\subsection{Algorithm}

\subsubsection{Normalization}
Before the estimate, we need to normalize the input data by subtracting the \textbf{mean} and dividing by the \textbf{variance}

\subsubsection{Estimate}
The estimate is computed as follows
$$ \hat{y} = \sigma(w^Tx + b) $$
where $\sigma$ is defined as
$$ \sigma(z) = \frac{1}{1 + e^-z} \in (0,1) $$

\subsubsection{Gradient}

$$ \nabla w = \frac{1}{m}x^T(\hat{y} - y) $$
$$\nabla b = \frac{1}{m}\sum(\hat{y} - y) b $$

\subsubsection{Update}


$$ w' = w - \mu \nabla w $$
$$ b' = b - \mu \nabla b $$

\subsubsection{Loss}
For the loss, we used the \textbf{Binary Cross Entropy} function, also called \textbf{Log Loss}.
It is defined as
$$ loss(\hat{y}, y) = -\frac{1}{n}(y log(\hat{y}) + (1-y)log(1-\hat{y})) $$

\subsubsection{Regularization}
Regularization is a technique used to prevent the overfittings. A regularization term is added to the optimization problem (i.e. the gradient calculation) to avoid overfitting.
The used version is called \textbf{L2}, also known as \textbf{Ridge Regression}. MIGLIORARE + BIBLIO\\

The regularization factor for the loss is defined as
$$ L2 = \frac{\lambda}{2}||w||^2 $$
where $L2$ is calculated as $ \frac{\lambda}{2}||w||^2 = \frac{\lambda}2{\displaystyle\sum_{j=1}^m w_j^2} $ \\
DA VEDERE

The loss then becomes
$$ loss(\hat{y}, y) = -\frac{1}{n}(y log(\hat{y}) + (1-y)log(1-\hat{y})) + \frac{\lambda}{2}||w||^2 $$

While the weights and bias gradients formula becomes
$$ \nabla w = \frac{1}{m}x^T(\hat{y} - y) + \lambda w $$
and
$$ \nabla b = \frac{1}{m}\sum(\hat{y} - y) + \lambda b$$

DA VEDERE

\subsection{Differences with Sklearn implementation}

In the following chapters the presented model performances will be compared with the Sklearn implementation, that has quite some differences.\\
First, the sklearn version doesn't use the SGD solver, it uses \textbf{L-BFGS-B - Software for Large-scale Bound-constrained Optimization} instead, by default.
% http://users.iems.northwestern.edu/~nocedal/lbfgsb.html
For this reason, the solver doesn't need any form of Learning Rate.\\
Also, with this implementation, the L2 Regularization is enabled by default as well.

% https://medium.com/@aditya97p/l1-and-l2-regularization-237438a9caa6
% https://github.com/mag3141592/LogisticRegression/blob/master/L2RegularizedLogisticRegression.py
% https://satishkumarmoparthi.medium.com/logistic-regression-with-l2-regularization-using-sgd-from-scratch-893692b48362
% https://towardsdatascience.com/batch-mini-batch-and-stochastic-gradient-descent-for-linear-regression-9fe4eefa637c


%----------------------------------------------------------------------------------------
%	PERFORMANCES 
%----------------------------------------------------------------------------------------

\section{Performances}

- come scalano le operazioni nella teoria
- come scalano le operazioni di numpy effettuate

\subsection{Preprocessing Performance}

Due to the limitations of the hardware used during the experiments, both locally and on Google Colab, we couldn't obtain significant improvements over preprocessing speed using PySpark, because of the lack of a high number of worker nodes.\\

The table below summarizes the time required by the Colab enviroment to complete various parts of the preprocessing section.
The numbers represents seconds of execution.

\begin{center}
	\begin{tabular}{ |c|c|c|c|c|c| }
		\hline
		Section      & Pandas & PySpark Single Node & PySpark Multiple Nodes \\
		\hline
		Data reading & 0      & 0                   & 0                      \\
		Common       & 0      & 0                   & 0                      \\
		Row removal  & 0      & 0                   & 0                      \\
		Balancing    & 0      & 0                   & 0                      \\
		Splitting    & 0      & 0                   & 0                      \\
		Total        & 0      & 0                   & 0                      \\

		\hline
	\end{tabular}
	\captionof{table}{Performance comparison between Pandas preprocessing and distributed PySpark preprocessing}
\end{center}

%----------------------------------------------------------------------------------------
%	EXPERIMENTS
%----------------------------------------------------------------------------------------

\section{Experiments}

Premessa su dataset

\subsection{Canceled Flights}

\subsubsection{Changing the Learning Rate}

The following experiment consists on the training and evaluation of the model with different values of the Learning Rate. The training is fixed at 100 iterations.

\begin{center}
	\begin{tabular}{ |c|c|c| }
		\hline
		LR & Test Loss & Train Loss \\
		\hline
		10 & 0         & 0          \\
		0  & 0         & 0          \\
		0  & 0         & 0          \\
		0  & 0         & 0          \\
		\hline
	\end{tabular}
	\captionof{table}{Your caption here}
\end{center}



\subsubsection{Implementing Mini-Batch}

\begin{center}
	\begin{tabular}{ |c|c|c| }
		\hline
		Batch Size & Test Loss & Train Loss \\
		\hline
		1          & 0         & 0          \\
		20         & 0         & 0          \\
		1000       & 0         & 0          \\
		Max        & 0         & 0          \\
		\hline
	\end{tabular}
	\captionof{table}{Your caption here}
\end{center}

\subsubsection{Adding the L2 regularization}

\begin{center}
	\begin{tabular}{ |c|c|c| }
		\hline
		L2   & Test Loss & Train Loss \\
		\hline
		1    & 0         & 0          \\
		20   & 0         & 0          \\
		1000 & 0         & 0          \\
		\hline
	\end{tabular}
	\captionof{table}{Your caption here}
\end{center}

\subsubsection{More iterations}

\begin{center}
	\begin{tabular}{ |c|c|c| }
		\hline
		Iterations & Test Loss & Train Loss \\
		\hline
		100        & 0         & 0          \\
		200        & 0         & 0          \\
		1000       & 0         & 0          \\
		\hline
	\end{tabular}
	\captionof{table}{Your caption here}
\end{center}

\subsubsection{Comparison with Sklearn}

\begin{center}
	\begin{tabular}{ |c|c|c|c|c|c|c| }
		\hline
		Model   & Iterations & LR & Batch Size & L2 & Test Loss & Train Loss \\
		\hline
		Custom  & 100        & 0  & 0          & 0  & 0         & 0          \\
		Sklearn & 100        & 0  & 0          & 0  & 0         & 0          \\
		\hline
	\end{tabular}
	\captionof{table}{Your caption here}
\end{center}

\subsection{Diverted Flights}

\subsubsection{Changing the Learning Rate}

\begin{center}
	\begin{tabular}{ |c|c|c|c|c| }
		\hline
		Iterations & LR & L2 & Test Loss & Train Loss \\
		\hline
		100        & 0  & 0  & 0         & 0          \\
		500        & 0  & 0  & 0         & 0          \\
		1000       & 0  & 0  & 0         & 0          \\
		\hline
	\end{tabular}
	\captionof{table}{Your caption here}
\end{center}

\subsubsection{Adding the L2 regularization}

\begin{center}
	\begin{tabular}{ |c|c|c|c|c| }
		\hline
		Iterations & LR & L2 & Test Loss & Train Loss \\
		\hline
		100        & 0  & 0  & 0         & 0          \\
		500        & 0  & 0  & 0         & 0          \\
		1000       & 0  & 0  & 0         & 0          \\
		\hline
	\end{tabular}
	\captionof{table}{Your caption here}
\end{center}

\subsubsection{Comparison with Sklearn}

\begin{center}
	\begin{tabular}{ |c|c|c|c|c| }
		\hline
		Iterations & LR & L2 & Test Loss & Train Loss \\
		\hline
		100        & 0  & 0  & 0         & 0          \\
		500        & 0  & 0  & 0         & 0          \\
		1000       & 0  & 0  & 0         & 0          \\
		\hline
	\end{tabular}
	\captionof{table}{Your caption here}
\end{center}


%----------------------------------------------------------------------------------------
%	RESULTS AND CONCLUSIONS
%----------------------------------------------------------------------------------------

\section{Results and Conclusions}

The reason of this experiments was to reproduce a complete flow to create and train a Logistic Regression model to classify flights data.\\
The dataset preprocessing has been successfully parallelized using PySpark, but the lack of a proper distributed system to test on lead to worse performances compared to the pandas implementation.
Instead, the model built using SGD, L2 Regularization, Normalization and Batching (properly tuned) resulted in an accuracy similar to the state-of-art Sklearn implementation.

%----------------------------------------------------------------------------------------
%	BIBLIOGRAPHY
%----------------------------------------------------------------------------------------

\printbibliography % Output the bibliography

% Logistic Regression
% Spark/PySpark
% Kaggle Dataset
% Mini Batch
% L2 Regularization
% Normalization
% SGD
% Il solver di Sklearn
% sklearn, numpy, pandas
% binary cross entropy
% sigmoid

%----------------------------------------------------------------------------------------

\end{document}