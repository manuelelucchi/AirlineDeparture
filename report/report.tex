%----------------------------------------------------------------------------------------
%	PACKAGES AND DOCUMENT CONFIGURATIONS
%----------------------------------------------------------------------------------------

\documentclass[
	letterpaper, % Paper size, specify a4paper (A4) or letterpaper (US letter)
	10pt, % Default font size, specify 10pt, 11pt or 12pt
]{class}

\addbibresource{bibliography.bib} % Bibliography file (located in the same folder as the template)

%----------------------------------------------------------------------------------------
%	REPORT INFORMATION
%----------------------------------------------------------------------------------------

\title{Airline Departure\\Data Analysis and Regression} % Report title

\author{Lucchi Manuele \& Tricella Davide} % Author name(s), add additional authors like: '\& James \textsc{Smith}'

\date{\today} % Date of the report

%----------------------------------------------------------------------------------------

\begin{document}

\maketitle % Insert the title, author and date using the information specified above

\begin{center}
	\begin{tabular}{l r}
		Instructors: Professor \textsc{Cesa-Bianchi} \& Professor \textsc{Malchiodi}
	\end{tabular}
\end{center}

%----------------------------------------------------------------------------------------
%	ABSTRACT
%----------------------------------------------------------------------------------------

\begin{abstract}
	The purpose of this paper is to evaluate the usage of a Logistic Regression model on a airlines dataset to predict flight cancellation or diversion, in a scalable and time/space efficient implementation.
\end{abstract}

\section{Definitions}\label{definitions} % Labels provide a point for referencing, in this case with \ref{definitions} to refer to this subsection number

\begin{description}
	\item[Stoichiometry] The relationship between the relative quantities of substances taking part in a reaction or forming a compound, typically a ratio of whole integers.
	\item[Atomic mass] The mass of an atom of a chemical element expressed in atomic mass units. It is approximately equivalent to the number of protons and neutrons in the atom (the mass number) or to the average number allowing for the relative abundances of different isotopes.
\end{description}

%----------------------------------------------------------------------------------------
%	DATASET
%----------------------------------------------------------------------------------------

\section{Dataset}

- Overview dataset
- Struttura delle colonne
- Divisione dei file
- Numero di record
- Quali colonne prendiamo
- Quanti record prendiamo

%----------------------------------------------------------------------------------------
%	PREPROCESSING TECHNIQUES
%----------------------------------------------------------------------------------------

\section{Preprocessing Techniques}

- balancing
- normalization
- splitting
- dates and enums to numbers

%----------------------------------------------------------------------------------------
%	PREPROCESSING PARALLELIZATION 
%----------------------------------------------------------------------------------------

\section{Preprocessing Parallelization}

%----------------------------------------------------------------------------------------
%	MODEL
%----------------------------------------------------------------------------------------

\section{Model}

%----------------------------------------------------------------------------------------
%	PERFORMANCES 
%----------------------------------------------------------------------------------------

\section{Performances}

%----------------------------------------------------------------------------------------
%	EXPERIMENTS
%----------------------------------------------------------------------------------------

\section{Experiments}

%----------------------------------------------------------------------------------------
%	RESULTS AND CONCLUSIONS
%----------------------------------------------------------------------------------------

\section{Results and Conclusions}

The atomic weight of magnesium is concluded to be \SI{24}{\gram\per\mol}, as determined by the stoichiometry of its chemical combination with oxygen. This result is in agreement with the accepted value.

\begin{figure}[H] % [H] forces the figure to be placed exactly where it appears in the text
	\centering % Horizontally center the figure
	%\includegraphics[width=0.65\textwidth]{placeholder} % Include the figure
	\caption{Figure caption.}
\end{figure}

%----------------------------------------------------------------------------------------
%	BIBLIOGRAPHY
%----------------------------------------------------------------------------------------

\printbibliography % Output the bibliography

%----------------------------------------------------------------------------------------

\end{document}