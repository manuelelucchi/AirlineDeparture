%%%%%%%%%%%%%%%%%%%%%%%%%%%%%%%%%%%%%%%%%
% University/School Laboratory Report
% LaTeX Template
% Version 4.0 (March 21, 2022)
%
% This template originates from:
% https://www.LaTeXTemplates.com
%
% Authors:
% Vel (vel@latextemplates.com)
% Linux and Unix Users Group at Virginia Tech Wiki
%
% License:
% CC BY-NC-SA 4.0 (https://creativecommons.org/licenses/by-nc-sa/4.0/)
%
%%%%%%%%%%%%%%%%%%%%%%%%%%%%%%%%%%%%%%%%%

%----------------------------------------------------------------------------------------
%	PACKAGES AND DOCUMENT CONFIGURATIONS
%----------------------------------------------------------------------------------------

\documentclass[
	letterpaper, % Paper size, specify a4paper (A4) or letterpaper (US letter)
	10pt, % Default font size, specify 10pt, 11pt or 12pt
]{class}

\addbibresource{bibliography.bib} % Bibliography file (located in the same folder as the template)

%----------------------------------------------------------------------------------------
%	REPORT INFORMATION
%----------------------------------------------------------------------------------------

\title{Determination of the Atomic \\ Weight of Magnesium \\ CHEM 101} % Report title

\author{Lucchi Manuele \& Tricella Davide} % Author name(s), add additional authors like: '\& James \textsc{Smith}'

\date{\today} % Date of the report

%----------------------------------------------------------------------------------------

\begin{document}

\maketitle % Insert the title, author and date using the information specified above

\begin{center}
	\begin{tabular}{l r}
		Date Performed: & February 13, 2022         \\ % Date the experiment was performed
		Partners:       & Cecilia \textsc{Smith}    \\ % Partner names
		                & Tajel \textsc{Khumalo}    \\
		Instructor:     & Professor \textsc{Rivera} % Instructor/supervisor
	\end{tabular}
\end{center}

%----------------------------------------------------------------------------------------
%	ABSTRACT
%----------------------------------------------------------------------------------------

\begin{abstract}
	Abstract text
\end{abstract}

%----------------------------------------------------------------------------------------
%	OBJECTIVE
%----------------------------------------------------------------------------------------

\section{Objective}

To determine the atomic weight of magnesium via its reaction with oxygen and to study the stoichiometry of the reaction (as defined in \ref{definitions}):

\begin{center}
	\ce{2 Mg + O2 -> 2 MgO} % Chemical equations entered in \ce{} commands, see the mhchem package documentation for more information
\end{center}

% If you have more than one objective, uncomment the below:
%\begin{description}
%	\item[First Objective] \hfill \\
%	Objective 1 text
%	\item[Second Objective] \hfill \\
%	Objective 2 text
%\end{description}

\subsection{Definitions}\label{definitions} % Labels provide a point for referencing, in this case with \ref{definitions} to refer to this subsection number

\begin{description}
	\item[Stoichiometry] The relationship between the relative quantities of substances taking part in a reaction or forming a compound, typically a ratio of whole integers.
	\item[Atomic mass] The mass of an atom of a chemical element expressed in atomic mass units. It is approximately equivalent to the number of protons and neutrons in the atom (the mass number) or to the average number allowing for the relative abundances of different isotopes.
\end{description}

%----------------------------------------------------------------------------------------
%	DATASET
%----------------------------------------------------------------------------------------

\section{Dataset}

\begin{tabular}{l l}
	Mass of empty crucible                             & \SI{7.28}{\gram} \\ % Scientific/technical units are output using the \SI command, see the siunitx package documentation for more information on how to use this command
	Mass of crucible and magnesium before heating      & \SI{8.59}{\gram} \\
	Mass of crucible and magnesium oxide after heating & \SI{9.46}{\gram} \\
	Balance used                                       & \#4              \\
	Magnesium from sample bottle                       & \#1
\end{tabular}

%----------------------------------------------------------------------------------------
%	PREPROCESSING TECHNIQUES
%----------------------------------------------------------------------------------------

\section{Preprocessing Techniques}

\begin{tabular}{ll}
	Mass of magnesium metal & = \SI{8.59}{\gram} - \SI{7.28}{\gram} \\
	                        & = \SI{1.31}{\gram}                    \\
	Mass of magnesium oxide & = \SI{9.46}{\gram} - \SI{7.28}{\gram} \\
	                        & = \SI{2.18}{\gram}                    \\
	Mass of oxygen          & = \SI{2.18}{\gram} - \SI{1.31}{\gram} \\
	                        & = \SI{0.87}{\gram}
\end{tabular}

Because of this reaction, the required ratio is the atomic weight of magnesium: \SI{16.00}{\gram} of oxygen as experimental mass of Mg: experimental mass of oxygen or $\frac{x}{1.31} = \frac{16}{0.87}$ from which, $M_{\ce{Mg}} = 16.00 \times \frac{1.31}{0.87} = 24.1 = \SI{24}{\gram\per\mole}$ (to two significant figures).

%----------------------------------------------------------------------------------------
%	PREPROCESSING PARALLELIZATION 
%----------------------------------------------------------------------------------------

\section{Preprocessing Parallelization}

%----------------------------------------------------------------------------------------
%	MODEL
%----------------------------------------------------------------------------------------

\section{Model}

%----------------------------------------------------------------------------------------
%	PERFORMANCES 
%----------------------------------------------------------------------------------------

\section{Performances}

%----------------------------------------------------------------------------------------
%	EXPERIMENTS
%----------------------------------------------------------------------------------------

\section{Experiments}

%----------------------------------------------------------------------------------------
%	RESULTS AND CONCLUSIONS
%----------------------------------------------------------------------------------------

\section{Results and Conclusions}

The atomic weight of magnesium is concluded to be \SI{24}{\gram\per\mol}, as determined by the stoichiometry of its chemical combination with oxygen. This result is in agreement with the accepted value.

\begin{figure}[H] % [H] forces the figure to be placed exactly where it appears in the text
	\centering % Horizontally center the figure
	%\includegraphics[width=0.65\textwidth]{placeholder} % Include the figure
	\caption{Figure caption.}
\end{figure}

%----------------------------------------------------------------------------------------
%	BIBLIOGRAPHY
%----------------------------------------------------------------------------------------

\printbibliography % Output the bibliography

%----------------------------------------------------------------------------------------

\end{document}